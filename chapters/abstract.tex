\chapter*{Abstract}
The study explores two concepts to devise a recommendation system for Galaxy. One idea is to find similar tools for each tool and another is to predict a set of possible next tools in workflows.

To find similarities among tools, we need to extract information from each tool under multiple categories like a tool's name, description, input and output file types and helptext. We take into account these categories one by one and compute similarity matrices. Each row in a similarity matrix keeps similarity scores of one tool against all other tools. These similarity scores depend on the similarity measure used to compute the score between a pair of tools. We compute three such similarity matrices, one each for input and output file types, name and description and helptext. To combine these matrices, one simple solution is to compute an average. But, assigning equal importance weights to each matrix might be sub-optimal. To find an optimal combination, we use optimization to learn importance weights for the corresponding rows for each tool in the similarity matrices. To define a loss function, we use a true similarity value based on the similarity measures. We take an array of 1.0 as true value because we use jaccard index and cosine similarity as similarity measures which give a positive real number between 0 and 1.

Next task analyzes workflows to predict a set of next tools at each stage of creating workflows. While creating workflows, it would be convenient to leaf through a set of next possible tools as a guide. It can assist the less experienced (Galaxy) users in creating workflows when they are unsure about which tools can further be joined. In addition, it can curtail the time taken in creating a workflow. To achieve that, we need to learn the connections among tools in order to be able to predict the next possible ones. To preprocess each workflow, we compute all the paths bridging the starting and ending tools. These paths contain a set of connected tools. We follow a classification approach to predict the next tools. For the classification task, we use a variant of neural networks (long short-term memory). It performs well for learning long range dependencies (tools connections) \cite{LiptonKEW15, SakSB14}. We report the accuracy as precision.


\chapter*{Zusammenfassung}
