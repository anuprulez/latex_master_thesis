\chapter*{Abstract}
The study explores two concepts to devise a recommendation system for Galaxy{\footnote{\url{https://usegalaxy.eu/}}} which include finding similar tools{\footnote{\url{https://galaxyproject.org/tools/}}} and predicting next tools in workflows{\footnote{\url{https://galaxyproject.org/learn/advanced-workflow/}}}. The recommendation system can apprise a Galaxy user of the extent to which the tools are related to one another in terms of their functions and types. To collect data about tools, we extract information from the name, description, input and output types and help text attributes. We compute vectors \cite{Foltz1996, DBLP:journals/corr/LeM14} for each tool using data from these attributes and apply similarity measures (jaccard index and cosine similarity) to estimate distance between each pair of vectors and form similarity matrix for each tool attribute. Each row in the similarity matrix holds similarity scores of one tool with all other tools. To combine these similarity scores, one solution is to compute an average. We use optimization \cite{KaoudiQTCA17} to learn optimal importance weights for the corresponding rows of a tool in the similarity matrices. To define a loss function for optimization, we use a true similarity value based on the similarity measures. Exhibiting an array of next possible tools at each step of picking one while creating workflows can be a meaningful addition to the recommendation system. It would be convenient to leaf through a set of next possible tools as a guide while creating workflows. It can assist the less experienced users if they are indecisive about the next tools. In addition, it can curtail the time taken in creating a workflow. To achieve that, we figure out all the unique paths bridging the input and output tools in the workflows to make them accessible to the downstream machine learning algorithms. We learn the connections among tools to be able to predict the next possible ones based on the previous connections in the path. To learn these connections, we follow a classification approach and use LSTM (long short-term memory), a variant of recurrent neural networks. This network performs well for long-range and sequential (tools connections) \cite{LiptonKEW15, SakSB14} data. We report the accuracy as precision.

\chapter*{Zusammenfassung}


