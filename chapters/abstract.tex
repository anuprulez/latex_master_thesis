\chapter*{Abstract}
The study explores two concepts to devise a recommendation system for Galaxy. One idea is to find similar Galaxy tools for each tool and another is to predict a set of possible next tools in Galaxy workflows.

To find similarities among tools, we need to extract information about each tool from its attributes like name, description, input and output file types and help text. We take into account these attributes one by one and compute similarity matrices. We compute three similarity matrices, one each for input and output file, name and description and help text attributes. Each row in a similarity matrix holds similarity scores of one tool against all the other tools. These similarity scores depend on the similarity measures (jaccard index and cosine similarity) used to compute the score between a pair of tools. To combine these matrices, one simple solution is to compute an average. But, assigning equal importance weight to each matrix might be sub-optimal. To find an optimal combination, we use optimization to learn importance weights for the corresponding rows for each tool in the similarity matrices. To define a loss function for optimization, we use a true similarity value based on the similarity measures. The similarity scores are positive real numbers between $0$ and $1$. We take an array of $1.0$ as the true value.

Next task analyzes workflows to predict a set of next tools at each stage of creating workflows. While creating workflows, it would be convenient to leaf through a set of next possible tools as a guide. It can assist the less experienced (Galaxy) users in creating workflows when they are unsure about which tools can further be connected. In addition, it can curtail the time taken in creating a workflow. To achieve that, we need to learn the connections among tools in order to be able to predict the next possible ones based on the previously connected tools. To preprocess the workflows to make them usable by downstream machine learning algorithms, we compute all the paths bridging the starting and end tools in all workflows. We follow a classification approach to predict the next tools and use LSTM (long short-term memory), a variant of recurrent neural networks. It performs well for learning long range, sequential and time-dependent data (tools connections) \cite{LiptonKEW15, SakSB14}. We report the accuracy as precision.


\chapter*{Zusammenfassung}
